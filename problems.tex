\section{Задачи}

В этом разделе размещены мои решения некоторых задач из списка Михаила Геннадьевича. Символом <<$\circ$>> помечены задачи, которые Михаил Геннадьевич считает обязательными (не уверен, что здесь появятся какие-то другие). То есть, это те же задачи, что помечены кружочком и у него. Нумерация задач также совпадает с Михаилом Геннадьевичем.

\begin{problem}[1$^\circ$]
    (А) Пусть $f \in R[a; b]$, $\{T^n\xi^n\}_{n = 1}^\infty$ --- последовательность отмеченных разбиений отрезка $[a; b]$, причём $d(T^n) \to 0$ при $n \to \infty$. Докажите, что
    \[
        \int\limits_a^bf(x)dx = \lim_{n \to \infty}\mathcal{S}(f, T^n\xi^n).
    \]

    (Б) Разобъём отрезок $[a; b]$ на $n$ равных отрезков $\Delta_j^{(n)}$, $1 \leqslant j \leqslant n$ длины $(b - a) / n$, а затем в каждом из них произвольным образом выберем метку $\xi_j^{(n)}$. Покажите, что
    \[
        \int\limits_a^bf(x)dx = \lim_{n \to \infty}\frac{b - a}{n}\sum_{j = 1}^nf(\xi_j^{(n)}).
    \]
\end{problem}

\begin{solution}
    (А) По определению предела последовательности $d(T^n)$,
    \[
        \forall \varepsilon > 0\;\exists \mathcal{N}_\varepsilon: (n > \mathcal{N}_\varepsilon) \Rightarrow \br{\abs{d(T^n)} < \varepsilon}\eqno(\ast)
    \]
    По определению интеграла Римана
    \[
        \forall \varepsilon > 0\; \exists \delta > 0: \br{d(T) < \delta} \Rightarrow \br{\abs{\mathcal{S}(f, T\xi) - I} < \varepsilon}.
    \]
    Перепишем последнее высказывание с учётом $(\ast)$:
    \[
        \forall\varepsilon > 0\; \exists N_\varepsilon \vcentcolon = \mathcal{N}_\varepsilon: (n > N_\varepsilon) \Rightarrow (\abs{\mathcal{S}(f, T\xi) - I} < \varepsilon).
    \]
    Значит, утверждение теоремы верно по определению предела последовательности.

    (Б) По предыдущему пункту
    \[
        \int\limits_a^bf(x)dx = \lim_{n \to \infty}\mathcal{S}(f, T^n\xi^n) = \lim_{n \to \infty}\br{\sum_{j = 1}^nf(\xi_j^{(n)})|\Delta_j^{(n)}|} = \lim_{n \to \infty}\frac{b - a}{n}\sum_{j = 1}^nf(\xi_j^{(n)}).
    \]
\end{solution}

\begin{problem}[2$^\circ$]
    Вычислите с помощью формулы из задачи 1.Б интегралы $\ds\int\limits_0^bx^2dx$ и $\ds\int\limits_0^bx^3dx$. В качестве $\xi_j^n$ возьмите правые концы соответствующих отрезков разбиения.
\end{problem}

\begin{solution}
    В разбиении на равные отрезки длины $n$ отрезка $[a; b]$ правый конец $j$-го отрезка будет иметь координату
    \[
        \frac{b - a}{n} \cdot j + a = \frac{b \cdot j + (n - j) \cdot a}{n}.
    \]
    Однако с учётом $a = 0$ (в нашей задаче), получаем $\xi_j^{(n)} = j \cdot b / n$. Пользуясь задачей 1, находим первый интеграл:
    \begin{multline*}
        \int\limits_0^bx^2dx = \lim_{n \to \infty}\frac{b}{n}\sum_{j = 1}^n\br{\frac{b \cdot j}{n}}^2 = \lim_{n \to \infty}\br{\frac{b}{n}}^3\frac{n(n + 1)(2n + 1)}{6} =\\ =\frac{b^3}{6}\lim_{n \to \infty}\frac{n(n + 1)(2n + 1)}{n^3} = \frac{b^3}{6}\lim_{n \to \infty}\br{1 + \frac{1}{n}}\br{2 + \frac{1}{n}} = \fbox{$\ds\frac{b^3}{3}$}
    \end{multline*}

    Теперь найдём второй интеграл:
    \[
        \int\limits_0^bx^3dx = \lim_{n \to \infty}\frac{b}{n}\sum_{j = 1}^n\br{\frac{b \cdot j}{n}}^3 = \lim_{n \to \infty}\br{\frac{b}{n}}^4\frac{n^2(n + 1)^2}{4} = \frac{b^4}{4}\lim_{n \to \infty}\br{1 + \frac{2}{n} + \frac{1}{n^2}} = \fbox{$\ds\frac{b^4}{4}$}
    \]
\end{solution}

\begin{problem}[4$^\circ$]
    Докажите, что не более чем счётное объединение множеств меры нуль по Лебегу имеет меру нуль по Лебегу. Покажите, что не более чем счётное множество имеет меру нуль по Лебегу.
\end{problem}

\begin{solution}
    Докажем первое утверждение. Пусть $\{E_i : i \in I\}$ --- система множеств меры нуль по Лебегу. Возьмём произвольное $\varepsilon > 0$ и найдём для каждого множества $E_i$ ($i \in I$) такое покрытие отрезками $\{\Delta^{(i)}_j : j \in J\}$, что $\sum\limits_{j \in J}\abs{\Delta^{(i)}_j} < \varepsilon / 2^{i + 1}$. Полученная система отрезков $\{\Delta^{(i)}_j : i \in I, j \in J\}$ образует покрытие множества $\bigcup\limits_{i \in I}E_i$. Просуммировав по всем $i$ и $j$, получим
    \[
        \sum_{i \in I}\sum_{j \in J}\abs{\Delta^{(i)}_j} \leqslant \frac{1}{2}\br{\frac{\varepsilon}{2} + \frac{\varepsilon}{4} + \ldots} \leqslant \frac{\varepsilon}{2} < \varepsilon.
    \]
    Согласно определению, множество $\bigcup\limits_{i \in I}E_i$ имеет меру нуль по Лебегу.

    Второе утверждение сведём к первому. Если множество $E$ не более чем счётно, то его элементы можно занумеровать натуральными числами: $E = \{e_i : i \in I\}$ ($I$ --- не более чем счётное множество индексов). Записав по-другому, получим $E = \bigcup\limits_{i \in I}\{e_i\}$. Множества $\{e_i\}$, очевидно, имеют меру нуль по Лебегу, поэтому (согласно первому утверждению) и $E$ имеет меру нуль по Лебегу.
\end{solution}

\begin{problem}[6$^\circ$]
    Построим \textit{канторовское троичное множество} $F_{1 / 3}$. Из отрезка $[0; 1]$ вырежем среднюю треть --- интервал $\br{\frac{1}{3}; \frac{2}{3}}$. Затем из двух оставшихся отрезков $\sqbr{0; \frac{1}{3}}$ и $\sqbr{\frac{2}{3}; 1}$ снова вырезаем интервал-среднюю треть. Повторим процедуру счётное число раз и получим после всех вырезаний множество $F_{1 / 3}$. Покажите, что это множество замкнуто и меры нуль по Лебегу.
\end{problem}

\begin{solution}
    Дополнение к нему есть объединение лучей $(-\infty; 1)$ и $(1; +\infty)$ и вырезанных интервалов. Объединение любого числа открытых множеств открыто, поэтому $\R \setminus F_{1 / 3}$ открыто, а значит, $F_{1 / 3}$ замкнуто.

    Заметим, что после $i$-го шага у нас получается $2^i$ отрезков длины $1 / 3^i$. Значит, суммарная длина отрезков после $i$-го шага равна $\ell_i \vcentcolon = (2 / 3)^i$. Отметим, что $\ell_i \to 0$ при $i \to \infty$, поэтому для любого $\varepsilon > 0$ существует $N \in \N$ такое, что $\ell_N < \varepsilon$. Значит, в качестве искомого множества отрезков можно взять отрезки, получающиеся на $N$-ом шаге алгоритма. Итак, нашли не более чем счётную систему отрезков, образующих покрытие $F_{1 / 3}$ и имеющих суммарную длину $< \varepsilon$. Значит, $F_{1 / 3}$ имеет меру нуль по Лебегу.
\end{solution}

\begin{problem}[11$^\circ$]
    Если $f \in R[a; b]$, то функцию $F: [a; b] \to \R$, $\ds F(x) \vcentcolon = \int\limits_x^bf(t)dt$ называют \textit{интегралом Римана с переменным нижним пределом}. Докажите, что $F \in C[a; b]$. Покажите, что если $f \in C(x)$, то $F \in D(x)$ и $F^\prime(x) = -f(x)$.
\end{problem}

\begin{solution}
    Решение почти дословно повторяет доказательства теорем 1 и 2 в вопросе 11.
\end{solution}

\begin{problem}[12$^\circ$]
    Приведите пример функции, которая интегрируема по Риману на некотором отрезке, но не имеет на нём первообразной.
\end{problem}

\begin{solution}
    Рассмотрим функцию $\sgn x$. Интегрируемость $\sgn$ по Риману на отрезке $[-1; 1]$ сразу следует из критерия Лебега. Пусть $\sgn$ имеет первообразную $F$ на $[-1; 1]$. Тогда она имеет вид
    \[
        F(x) =
        \begin{cases}
            -x + C_1,&\text{при $x < 0$},\\
            x + C_2,&\text{при $x \geqslant 0$},
        \end{cases}
    \]
    где $C_1, C_2 \in \R$. $F^\prime(x) = \sgn x$ для всех $x \ne 0$. Чтобы функция была дифференцируема в точке $x = 0$, она должна быть непрерывна в этой точке, а для этого должно выполняться $C_1 = C_2$, однако в таком случае $F(x) = \abs{x} + C$ ($C \in \R$), а она, как известно, не дифференцируема в точке $x = 0$.
\end{solution}

\begin{problem}[13$^\circ$]
    Приведите пример функции, которая на некотором отрезке имеет первообразную, но не интегрируема по Риману.
\end{problem}

\begin{solution}
    Рассмотрим функцию 
    \[
        F(x) \vcentcolon = 
        \begin{cases}
            x\sqrt{x}\sin\frac{1}{x},& x \ne 0,\\
            0,& x = 0.
        \end{cases}
    \]
    Эта функция дифференцируема на $(0; 1]$:
    \[
        \br{x^{3 / 2}\sin\frac{1}{x}}^\prime = \frac{3x\sin\frac{1}{x} - 2\cos\frac{1}{x}}{2\sqrt{x}}.
    \]
    Проверим дифференцируемость в точке $x = 0$:
    \[
        F^\prime(0) = \lim_{x \to 0}\frac{F(x) - F(0)}{x - 0} = \lim_{x \to 0}\sqrt{x}\sin\frac{1}{x} = 0.
    \]
    Последнее равенство верно в силу того, что $\sqrt{x}$ --- БМФ при $x \to 0$, а $\sin\frac{1}{x}$ --- ограниченная функция. Итак, обозначим $f(x) \vcentcolon = F^\prime(x)$. Эта функция неограничена на отрезке $[0; 1]$, а потому не интегрируема по Риману. В то же время, у неё есть первообразная $F$ на этом отрезке.
\end{solution}

\begin{problem}[15$^\circ$]
    Используя первую теорему о среднем для интеграла Римана, получите из остаточного члена формулы Тейлора в интегральной форме
    \[
        \frac{1}{n!}\int\limits_{x_0}^x(x - t)^nf^{(n + 1)}(t)dt
    \]
    остаточный член в форме Лагранжа
    \[
        \frac{f^{(n + 1)}(\xi)}{(n + 1)!}(x - x_0)^{n + 1},\quad\text{$\xi$ лежит между $x_0$ и $x$}.
    \]
\end{problem}

\begin{solution}
    Функция $f^{(n + 1)}(t)$ непрерывна (т.\,к. $f \in C^{(n + 1)}[a; b]$); применив первую теорему о среднем для интеграла Римана (см. вопрос 14), получаем
    \begin{multline*}
        \frac{1}{n!}\int\limits_{x_0}^x(x - t)^nf^{(n + 1)}(t)dt = \frac{1}{n!}f^{(n + 1)}(\xi)\int\limits_{x_0}^x(x - t)^ndt =\\ = -\frac{1}{n!}f^{(n + 1)}(\xi)\left.\frac{(x - t)^{n + 1}}{n + 1}\right|_{x_0}^x = \frac{f^{(n + 1)}}{(n + 1)!}(x - x_0)^{n + 1},
    \end{multline*}
    где $\xi$ лежит между $x_0$ и $x$.
\end{solution}

\begin{problem}[17$^\circ$]
    Вычислите $\V\limits_0^1f$, где
    \[
        f(x) =
        \begin{cases}
            1,& x = 0,\\
            \sin x,& 0 < x \leqslant 1.
        \end{cases}
    \]
\end{problem}

\begin{solution}
    Не понял, как делать.
\end{solution}

