\section{Задачи}

В этом разделе размещены мои решения некоторых задач из списка Михаила Геннадьевича. Символом <<$\circ$>> помечены задачи, которые Михаил Геннадьевич считает обязательными (не уверен, что здесь появятся какие-то другие). То есть, это те же задачи, что помечены кружочком и у него. Нумерация задач также совпадает с Михаилом Геннадьевичем.

\begin{problem}[1$^\circ$]
    (А) Пусть $f \in R[a; b]$, $\{T^n\xi^n\}_{n = 1}^\infty$ --- последовательность отмеченных разбиений отрезка $[a; b]$, причём $d(T^n) \to 0$ при $n \to \infty$. Докажите, что
    \[
        \int\limits_a^bf(x)dx = \lim_{n \to \infty}\mathcal{S}(f, T^n\xi^n).
    \]

    (Б) Разобъём отрезок $[a; b]$ на $n$ равных отрезков $\Delta_j^{(n)}$, $1 \leqslant j \leqslant n$ длины $(b - a) / n$, а затем в каждом из них произвольным образом выберем метку $\xi_j^{(n)}$. Покажите, что
    \[
        \int\limits_a^bf(x)dx = \lim_{n \to \infty}\frac{b - a}{n}\sum_{j = 1}^nf(\xi_j^{(n)}).
    \]
\end{problem}

\begin{solution}
    (А) По определению предела последовательности $d(T^n)$,
    \[
        \forall \varepsilon > 0\;\exists \mathcal{N}_\varepsilon: (n > \mathcal{N}_\varepsilon) \Rightarrow \br{\abs{d(T^n)} < \varepsilon}\eqno(\ast)
    \]
    По определению интеграла Римана
    \[
        \forall \varepsilon > 0\; \exists \delta > 0: \br{d(T) < \delta} \Rightarrow \br{\abs{\mathcal{S}(f, T\xi) - I} < \varepsilon}.
    \]
    Перепишем последнее высказывание с учётом $(\ast)$:
    \[
        \forall\varepsilon > 0\; \exists N_\varepsilon \vcentcolon = \mathcal{N}_\varepsilon: (n > N_\varepsilon) \Rightarrow (\abs{\mathcal{S}(f, T\xi) - I} < \varepsilon).
    \]
    Значит, утверждение теоремы верно по определению предела последовательности.

    (Б) По предыдущему пункту
    \[
        \int\limits_a^bf(x)dx = \lim_{n \to \infty}\mathcal{S}(f, T^n\xi^n) = \lim_{n \to \infty}\br{\sum_{j = 1}^nf(\xi_j^{(n)})|\Delta_j^{(n)}|} = \lim_{n \to \infty}\frac{b - a}{n}\sum_{j = 1}^nf(\xi_j^{(n)}).
    \]
\end{solution}

\begin{problem}[2$^\circ$]
    Вычислите с помощью формулы из задачи 1.Б интегралы $\ds\int\limits_0^bx^2dx$ и $\ds\int\limits_0^bx^3dx$. В качестве $\xi_j^n$ возьмите правые концы соответствующих отрезков разбиения.
\end{problem}

\begin{solution}
    В разбиении на равные отрезки длины $n$ отрезка $[a; b]$ правый конец $j$-го отрезка будет иметь координату
    \[
        \frac{b - a}{n} \cdot j + a = \frac{b \cdot j + (n - j) \cdot a}{n}.
    \]
    Однако с учётом $a = 0$ (в нашей задаче), получаем $\xi_j^{(n)} = j \cdot b / n$. Пользуясь задачей 1, находим первый интеграл:
    \begin{multline*}
        \int\limits_0^bx^2dx = \lim_{n \to \infty}\frac{b}{n}\sum_{j = 1}^n\br{\frac{b \cdot j}{n}}^2 = \lim_{n \to \infty}\br{\frac{b}{n}}^3\frac{n(n + 1)(2n + 1)}{6} =\\ =\frac{b^3}{6}\lim_{n \to \infty}\frac{n(n + 1)(2n + 1)}{n^3} = \frac{b^3}{6}\lim_{n \to \infty}\br{1 + \frac{1}{n}}\br{2 + \frac{1}{n}} = \fbox{$\ds\frac{b^3}{3}$}
    \end{multline*}

    Теперь найдём второй интеграл:
    \[
        \int\limits_0^bx^3dx = \lim_{n \to \infty}\frac{b}{n}\sum_{j = 1}^n\br{\frac{b \cdot j}{n}}^3 = \lim_{n \to \infty}\br{\frac{b}{n}}^4\frac{n^2(n + 1)^2}{4} = \frac{b^4}{4}\lim_{n \to \infty}\br{1 + \frac{2}{n} + \frac{1}{n^2}} = \fbox{$\ds\frac{b^4}{4}$}
    \]
\end{solution}

\begin{problem}[4$^\circ$]
    Докажите, что не более чем счётное объединение множеств меры нуль по Лебегу имеет меру нуль по Лебегу. Покажите, что не более чем счётное множество имеет меру нуль по Лебегу.
\end{problem}

\begin{solution}
    Докажем первое утверждение. Пусть $\{E_i : i \in I\}$ --- система множеств меры нуль по Лебегу. Возьмём произвольное $\varepsilon > 0$ и найдём для каждого множества $E_i$ ($i \in I$) такое покрытие отрезками $\{\Delta^{(i)}_j : j \in J\}$, что $\sum\limits_{j \in J}\abs{\Delta^{(i)}_j} < \varepsilon / 2^{i + 1}$. Полученная система отрезков $\{\Delta^{(i)}_j : i \in I, j \in J\}$ образует покрытие множества $\bigcup\limits_{i \in I}E_i$. Просуммировав по всем $i$ и $j$, получим
    \[
        \sum_{i \in I}\sum_{j \in J}\abs{\Delta^{(i)}_j} \leqslant \frac{1}{2}\br{\frac{\varepsilon}{2} + \frac{\varepsilon}{4} + \ldots} \leqslant \frac{\varepsilon}{2} < \varepsilon.
    \]
    Согласно определению, множество $\bigcup\limits_{i \in I}E_i$ имеет меру нуль по Лебегу.

    Второе утверждение сведём к первому. Если множество $E$ не более чем счётно, то его элементы можно занумеровать натуральными числами: $E = \{e_i : i \in I\}$ ($I$ --- не более чем счётное множество индексов). Записав по-другому, получим $E = \bigcup\limits_{i \in I}\{e_i\}$. Множества $\{e_i\}$, очевидно, имеют меру нуль по Лебегу, поэтому (согласно первому утверждению) и $E$ имеет меру нуль по Лебегу.
\end{solution}

\begin{problem}[6$^\circ$]
    Построим \textit{канторовское троичное множество} $F_{1 / 3}$. Из отрезка $[0; 1]$ вырежем среднюю треть --- интервал $\br{\frac{1}{3}; \frac{2}{3}}$. Затем из двух оставшихся отрезков $\sqbr{0; \frac{1}{3}}$ и $\sqbr{\frac{2}{3}; 1}$ снова вырезаем интервал-среднюю треть. Повторим процедуру счётное число раз и получим после всех вырезаний множество $F_{1 / 3}$. Покажите, что это множество замкнуто и меры нуль по Лебегу.
\end{problem}

\begin{solution}
    Дополнение к нему есть объединение лучей $(-\infty; 1)$ и $(1; +\infty)$ и вырезанных интервалов. Объединение любого числа открытых множеств открыто, поэтому $\R \setminus F_{1 / 3}$ открыто, а значит, $F_{1 / 3}$ замкнуто.

    Заметим, что после $i$-го шага у нас получается $2^i$ отрезков длины $1 / 3^i$. Значит, суммарная длина отрезков после $i$-го шага равна $\ell_i \vcentcolon = (2 / 3)^i$. Отметим, что $\ell_i \to 0$ при $i \to \infty$, поэтому для любого $\varepsilon > 0$ существует $N \in \N$ такое, что $\ell_N < \varepsilon$. Значит, в качестве искомого множества отрезков можно взять отрезки, получающиеся на $N$-ом шаге алгоритма. Итак, нашли не более чем счётную систему отрезков, образующих покрытие $F_{1 / 3}$ и имеющих суммарную длину $< \varepsilon$. Значит, $F_{1 / 3}$ имеет меру нуль по Лебегу.
\end{solution}

\begin{problem}[11$^\circ$]
    Если $f \in R[a; b]$, то функцию $F: [a; b] \to \R$, $\ds F(x) \vcentcolon = \int\limits_x^bf(t)dt$ называют \textit{интегралом Римана с переменным нижним пределом}. Докажите, что $F \in C[a; b]$. Покажите, что если $f \in C(x)$, то $F \in D(x)$ и $F^\prime(x) = -f(x)$.
\end{problem}

\begin{solution}
    Решение почти дословно повторяет доказательства теорем 1 и 2 в вопросе 11.
\end{solution}

\begin{problem}[12$^\circ$]
    Приведите пример функции, которая интегрируема по Риману на некотором отрезке, но не имеет на нём первообразной.
\end{problem}

\begin{solution}
    Рассмотрим функцию $\sgn x$. Интегрируемость $\sgn$ по Риману на отрезке $[-1; 1]$ сразу следует из критерия Лебега. Пусть $\sgn$ имеет первообразную $F$ на $[-1; 1]$. Тогда она имеет вид
    \[
        F(x) =
        \begin{cases}
            -x + C_1,&\text{при $x < 0$},\\
            x + C_2,&\text{при $x \geqslant 0$},
        \end{cases}
    \]
    где $C_1, C_2 \in \R$. $F^\prime(x) = \sgn x$ для всех $x \ne 0$. Чтобы функция была дифференцируема в точке $x = 0$, она должна быть непрерывна в этой точке, а для этого должно выполняться $C_1 = C_2$, однако в таком случае $F(x) = \abs{x} + C$ ($C \in \R$), а она, как известно, не дифференцируема в точке $x = 0$.
\end{solution}

\begin{problem}[13$^\circ$]
    Приведите пример функции, которая на некотором отрезке имеет первообразную, но не интегрируема по Риману.
\end{problem}

\begin{solution}
    Рассмотрим функцию 
    \[
        F(x) \vcentcolon = 
        \begin{cases}
            x\sqrt{x}\sin\frac{1}{x},& x \ne 0,\\
            0,& x = 0.
        \end{cases}
    \]
    Эта функция дифференцируема на $(0; 1]$:
    \[
        \br{x^{3 / 2}\sin\frac{1}{x}}^\prime = \frac{3x\sin\frac{1}{x} - 2\cos\frac{1}{x}}{2\sqrt{x}}.
    \]
    Проверим дифференцируемость в точке $x = 0$:
    \[
        F^\prime(0) = \lim_{x \to 0}\frac{F(x) - F(0)}{x - 0} = \lim_{x \to 0}\sqrt{x}\sin\frac{1}{x} = 0.
    \]
    Последнее равенство верно в силу того, что $\sqrt{x}$ --- БМФ при $x \to 0$, а $\sin\frac{1}{x}$ --- ограниченная функция. Итак, обозначим $f(x) \vcentcolon = F^\prime(x)$. Эта функция неограничена на отрезке $[0; 1]$, а потому не интегрируема по Риману. В то же время, у неё есть первообразная $F$ на этом отрезке.
\end{solution}

\begin{problem}[15$^\circ$]
    Используя первую теорему о среднем для интеграла Римана, получите из остаточного члена формулы Тейлора в интегральной форме
    \[
        \frac{1}{n!}\int\limits_{x_0}^x(x - t)^nf^{(n + 1)}(t)dt
    \]
    остаточный член в форме Лагранжа
    \[
        \frac{f^{(n + 1)}(\xi)}{(n + 1)!}(x - x_0)^{n + 1},\quad\text{$\xi$ лежит между $x_0$ и $x$}.
    \]
\end{problem}

\begin{solution}
    Функция $f^{(n + 1)}(t)$ непрерывна (т.\,к. $f \in C^{(n + 1)}[a; b]$); применив первую теорему о среднем для интеграла Римана (см. вопрос 14), получаем
    \begin{multline*}
        \frac{1}{n!}\int\limits_{x_0}^x(x - t)^nf^{(n + 1)}(t)dt = \frac{1}{n!}f^{(n + 1)}(\xi)\int\limits_{x_0}^x(x - t)^ndt =\\ = -\frac{1}{n!}f^{(n + 1)}(\xi)\left.\frac{(x - t)^{n + 1}}{n + 1}\right|_{x_0}^x = \frac{f^{(n + 1)}}{(n + 1)!}(x - x_0)^{n + 1},
    \end{multline*}
    где $\xi$ лежит между $x_0$ и $x$.
\end{solution}

\begin{problem}[17$^\circ$]
    Вычислите $\V\limits_0^1f$, где
    \[
        f(x) =
        \begin{cases}
            1,& x = 0,\\
            \sin x,& 0 < x \leqslant 1.
        \end{cases}
    \]
\end{problem}

\begin{solution}
    Здесь единственное отличие от теоремы о вариации монотонной функции в том, что первый модуль раскроется в другую сторону. Рассмотрим любое разбиение $T = \{[a_{i - 1}; a_i]\}_{i = 1}^m$
    \[
        V(f, T) = \sum_{i = 1}^m\abs{f(a_i) - f(a_{i - 1})} = f(a_0) - f(a_1) + \sum_{i = 2}^m\br{f(a_i) - f(a_{i - 1})} = 1 + \sin 1 - 2 \cdot \sin a_1.
    \]
    Отсюда
    \[
        \V_0^1f = \sup_T(1 + \sin 1 - 2 \cdot \sin a_1) = \lim_{a_1 \to 0+}(1 + \sin 1 - 2 \cdot \sin a_1) = 1 + \sin 1.
    \]
\end{solution}

\begin{problem}[18$^\circ$]
    Вычислите $\V\limits_0^1f$, где
    \[
        f(x) =
        \begin{cases}
            0,& x = 0,\\
            x\sin\br{\frac{1}{x}},& 0 < x \leqslant 1.
        \end{cases}
    \]
\end{problem}

\begin{solution}
    Это важный пример непрерывной на отрезке функции, вариация которой на этом отрезке неограничена. Возьмём в качестве границ отрезков разбиения точки касания графика функции $f$ с прямыми $y = x$ и $y = -x$. Точки находятся так:
    \[
        \sin\br{\frac{1}{x}} = \pm 1\qquad\Leftrightarrow\qquad\frac{1}{x} = \frac{\pi}{2} + \pi k,\quad k \in \N\qquad\Leftrightarrow\qquad x = \frac{1}{\pi} \cdot \frac{2}{k + 2},\quad k \in \N.
    \]

    Итак, рассматриваем разбиения вида $T = \{a_{i - 1}; a_i\}_{i = 1}^m$, где
    \[
        a_0 = 1,\quad a_i = \frac{1}{\pi} \cdot \frac{2}{i + 2},\quad a_m = 0,
    \]
    $i = 1, \ldots, m - 1$. Причём, $f(a_{2k}) = a_{2k}$ и $f(a_{2k + 1}) = -a_{2k + 1}$. Теперь
    \[
        V(f, T) = \sum_{i = 0}^m\abs{f(a_i) - f(a_{i - 1})} \geqslant \sum_{i = 1}^{m - 1}\abs{f(a_i) - f(a_{i - 1})} = \sum_{i = 1}^{m - 1}\br{\frac{1}{\pi} \cdot \frac{2}{i + 2} + \frac{1}{\pi} \cdot \frac{2}{i + 1}} = \frac{2}{\pi}\sum_{i = 1}^{m - 1}\frac{1}{i}.
    \]

    Справа стоит частичная сумма гармонического ряда, который, как известно, расходится. Поэтому вариация функции $f$ на отрезке $[0; 1]$ неограничена.
\end{solution}

\begin{problem}[19$^\circ$]
    Докажите, что функция Дирихле не является функцией ограниченной вариации ни на каком невырожденном в точку отрезке $[a; b]$.
\end{problem}

\begin{solution}
    На любом отрезке бесконечно много рациональных и иррациональных чисел. Поэтому в сумме из определения вариации мы сможем сделать сколько угодно слагаемых, равных $1$. Таким образом, вариация функции Дирихле неограничена.
\end{solution}

\begin{problem}[21$^\circ$]
    Докажите, что интеграл Римана с переменным верхним пределом от любой функции $f \in R[a; b]$ является функцией ограниченной вариации на $[a; b]$.
\end{problem}

\begin{solution}
    Рассмотрим произвольное разбиение $T = \{[a_{i - 1}; a_i]\}_{i = 1}^m$. Тогда
    \[
        V(f, T) = \sum_{i = 1}^m\abs{\int\limits_a^{a_i}f(t)dt - \int\limits_a^{a_{i - 1}}f(t)dt} = \sum_{i = 1}^m\abs{\,\int\limits_{a_{i - 1}}^{a_i}f(t)dt}.
    \]

    Число в правой части является конечным (по теореме об интегрируемости на подотрезках). Отсюда, $\ds\V\limits_a^b\int\limits_a^xf(t)dt < +\infty$, т.\,к. $f \in R[a; b]$.
\end{solution}

\begin{problem}[22$^\circ$]
    Докажите, что если $f \in BV[a; b]$, то $\alpha f \in BV[a; b]$ при любом $\alpha \in \R$ и
    \[
        \V_a^b(\alpha f) = \alpha \V_a^bf.
    \]
\end{problem}

\begin{solution}
    Рассмотрим произвольное разбиение $T = \{[a_{i - 1}; a_i]\}_{i = 1}^m$. Тогда
    \[
        V(\alpha f, T) = \sum_{i = 1}^m\abs{\alpha f(a_i) - \alpha f(a_{i - 1})} = \alpha\sum_{i = 1}^m\abs{f(a_i) - f(a_{i - 1})} = \alpha V(f, T).
    \]

    Отсюда $\V\limits_a^b(\alpha f) = \alpha \V\limits_a^bf < +\infty$, т.\,к. $f \in BV[a; b]$.
\end{solution}

\begin{problem}[23$^\circ$]
    Докажите, что если $f, g \in BV[a; b]$, то $f + g, fg \in BV[a; b]$, причём
    \[
        \V_a^b(f + g) \leqslant \V_a^bf + \V_a^bg,\qquad \V_a^b(fg) \leqslant M\br{\V_a^bf + \V_a^bg},\quad M \vcentcolon = \{f(x), g(x) : x \in [a; b]\}.
    \]
\end{problem}

\begin{solution}
    Рассмотрим произвольное разбиение $T = \{[a_{i - 1}; a_i]\}$. Тогда
    \begin{multline*}
        V(f + g, T) = \sum_{i = 1}^m\abs{f(a_i) + g(a_i) - f(a_{i - 1}) - g(a_{i - 1})} \leqslant\\ \leqslant \sum_{i = 1}^m\abs{f(a_i) - f(a_{i - 1})} + \sum_{i = 1}^m\abs{g(a_i) - g(a_{i - 1})} = V(f, T) + V(g, T)
    \end{multline*}
    и
    \begin{multline*}
        V(fg, t) = \sum_{i = 1}^m\abs{f(a_i)g(a_i) - f(a_{i - 1})g(a_{i - 1})} =\\ = \sum_{i = 1}^m\abs{f(a_i)g(a_i) - f(a_i)g(a_{i - 1}) + f(a_i)g(a_{i - 1}) - f(a_{i - 1})g(a_{i - 1})} =\\ = \sum_{i = 1}^m\abs{f(a_i)(g(a_i) - g(a_{i - 1})) + g(a_{i - 1})(f(a_i) - f(a_{i - 1}))} \leqslant M\br{\V\limits_a^bf + \V\limits_a^bg}.
    \end{multline*}

    Отсюда сразу следует требуемое.
\end{solution}

\begin{problem}[26$^\circ$]
    Вычислите интеграл Римана "---Стилтьеса $\ds(RS)\int\limits_{-1}^1dG(x)$,
    \[
        G(x) =
        \begin{cases}
            -1,&\text{если $x < 0$},\\
            1,&\text{если $x \geqslant 0$}.
        \end{cases}
    \]
\end{problem}

\begin{solution}
    Рассмотрим произвольное разбиение $T = \{[a_{i - 1}; a_i]\}_{i = 1}^m$ с любыми метками. Найдём наименьшее $i_0 \in [1; m]$, что $a_{i_0 - 1} < 0$, а $a_{i_0} \geqslant 0$. Распишем интегральную сумму Римана "---Стилтьеса по определению:
    \begin{multline*}
        \mathcal{S}(dG, T\xi) = \sum_{i = 1}^m\br{G(a_i) - G(a_{i - 1})} = \sum_{i < i_0}\underbrace{\br{G(a_i) - G(a_{i - 1})}}_{= 0} +\\ + \underbrace{G(a_{i_0}) - G(a_{i_0 - 1})}_{= 2} + \sum_{i > i_0}\underbrace{\br{G(a_i) - G(a_{i - 1})}}_{= 0} = 2.
    \end{multline*}

    Получили, что для всех разбиений эта сумма равна $2$. А значит, и интеграл равен $2$.
\end{solution}

\begin{problem}[27$^\circ$]
    Для любой функции $f \in C[-1; 1]$ и функции $G$ из предыдущей задачи вычислите $\ds(RS)\int\limits_{-1}^1f(x)dG(x)$.
\end{problem}

\begin{solution}
    Рассмотрим произвольное разбиение $T = \{[a_{i - 1}; a_i]\}_{i = 1}^m$ с любыми метками. Найдём наименьшее $i_0 \in [1; m]$, что $a_{i_0 - 1} < 0$, а $a_{i_0} \geqslant 0$. Распишем интегральную сумму Римана "---Стилтьеса по определению:
    \begin{multline*}
        \mathcal{S}(fdG, T\xi) = \sum_{i = 1}^mf(\xi_i)\br{G(a_i) - G(a_{i - 1})} = \sum_{i < i_0}f(\xi_i)\underbrace{\br{G(a_i) - G(a_{i - 1})}}_{= 0} +\\ + f(\xi_{i_0})\underbrace{G(a_{i_0}) - G(a_{i_0 - 1})}_{= 2} + \sum_{i > i_0}f(\xi_i)\underbrace{\br{G(a_i) - G(a_{i - 1})}}_{= 0} = 2f(\xi_{i_0}).
    \end{multline*}

    Возьмём произвольное $\varepsilon > 0$ и положим $I \vcentcolon = 2f(0)$. При любом разбиении мы можем взять $\varepsilon_{i_0} = 0$ (т.\,к. всегда найдётся отрезок разбиения, содержащий точку $0$) и для такого разбиения разность между интегральной суммой и $I$ будет $0 < \varepsilon$. Так что искомый интеграл равен $2f(0)$.
\end{solution}

\begin{problem}[28$^\circ$]
    Рассмотрим функции $f, G: [0; 1] \to \R$,
    \[
        f(x) =
        \begin{cases}
            1 & \text{при $x = \frac{1}{2}$},\\
            0 & \text{иначе},
        \end{cases}\quad
        G(x) =
        \begin{cases}
            0 & \text{при $x \leqslant \frac{1}{2}$},\\
            1 & \text{при $x > \frac{1}{2}$}.
        \end{cases}
    \]
    Покажите, что $f$ не интегрируема по $G$ по отрезку $[0; 1]$ в смысла Римана "---Стилтьеса.
\end{problem}

\begin{solution}
    Для любого разбиения можно взять метки $\{\xi_i\}_{i = 1}^m$, среди которых не найдётся $\xi_{i_0} = \frac{1}{2}$. При этом, для этого же разбиения можно взять другой набор меток $\{\xi^\prime_i\}_{i = 1}^m$, что среди них найдётся $\xi^\prime_{j_0} = \frac{1}{2}$. Для первого набора меток интеральная сумма Римана "---Стилтьеса будет равна $0$ (все $f(\xi_i)$ равны нулю), для второго --- $\frac{1}{2}$. Таким образом, ни для какого $\varepsilon < \frac{1}{2}$ не найдётся $\delta > 0$ такого, что любое $\delta$-разбиение удовлетворяет свойству из определения интеграла Римана "---Стилтьеса для какого-то $I$. Так что функция $f$ не интегрируема по Риману "---Стилтьесу по функции $G$ на отрезке $[0; 1]$.
\end{solution}

\begin{problem}[31$^\circ$]
    При каких действительных $\alpha$ сходится $\ds\int\limits_2^{+\infty}\frac{dx}{x\ln^\alpha x}$?
\end{problem}

\begin{solution}
    По определению несобственного интеграла:
    \[
        \int\limits_2^{+\infty}\frac{dx}{x\ln^\alpha x} = \lim_{b \to +\infty}\int\limits_2^{b}\frac{dx}{x\ln^\alpha x}.
    \]
    Найдём интеграл Римана под знаком предела:
    \[
        \int\limits_2^b\frac{dx}{x\ln^\alpha x} =
        \left\{
            \begin{array}{l}
                \ln x = t,\\
                x = e^t,\\
                dx = e^tdt
            \end{array}
        \right\} = \int\limits_2^b\frac{\cancel{e^t}dt}{\cancel{e^t} \cdot t^\alpha} = \int\limits_2^bt^{-\alpha}dt =
    \begin{cases}
        \left.\frac{t^{1 - \alpha}}{1 - \alpha}\right|_2^b = \frac{1}{1 - \alpha}\br{b^{1 - \alpha} - 2^{1 - \alpha}},&\alpha \ne 1\\
        \left.\ln t\;\right|_2^b = \ln b - \ln 2,&\alpha = 1.
    \end{cases}
    \]
    Отсюда видно, что при $\alpha \leqslant 1$ значение предела есть $+\infty$, а при $\alpha > 1$ интеграл сходится.
\end{solution}

\begin{problem}[32$^\circ$]
    Докажите сходимость несобственного интеграла
    \[
        \frac{1}{\sqrt{2\pi}}\int\limits_{-\infty}^{+\infty}e^{-\frac{x^2}{2}}dx.
    \]
\end{problem}

\begin{solution}
    Напишу позже.
\end{solution}

\begin{problem}[35$^\circ$]
    Докажите, что для дискретной метрики выполнены все аксиомы метрики.
\end{problem}

\begin{solution}
    Первые две аксиомы, очевидно, выполнены. Проверим неравенство треугольника. Если точки $x$, $y$ и $z$ совпадают, то оно выполнено. Иначе:
    \[
        \rho(x, y) \leqslant 1 \leqslant \rho(x, z) + \rho(z, y).
    \]
\end{solution}

\begin{problem}[36$^\circ$]
    Докажите, что для парижской метрики выполнены все аксиомы метрики.
\end{problem}

\begin{solution}
    Первые две аксиомы, очевидно, выполнены. Проверим неравенство треугольника. В случае совпадения некоторых точек среди $x$, $y$ и $z$ неравенство выполнено. Осталось рассмотреть случай, когда они попарно различны:
    \[
        \rho(x, z) + \rho(z, y) = f(x) + 2\underbrace{f(z)}_{\geqslant 0} + f(y) \geqslant f(x) + f(y) = \rho(x, y).
    \]
\end{solution}

\begin{problem}[37$^\circ$]
    Докажите, что метрика Хэмминга удовлетворяет всем аксиомам метрики.
\end{problem}

\begin{solution}
    Первые два свойства, очевидно, выполнены. Проверим неравенство треугольника.
    \[
        \rho(x, z) + \rho(z, y) = \sum_{i = 1}^n\abs{x_i - z_i} + \sum_{i = 1}^n\abs{z_i - y_i} \geqslant \sum_{i = 1}^n\abs{x_i - y_i} = \rho(x, y).
    \]
\end{solution}

\begin{problem}[39$^\circ$]
    Докажите, что пара $(C[a; b], \norm{\bs{\cdot}}_1)$, где $\norm{f}_1 \vcentcolon = \max\limits_{x \in [a; b]}\abs{f(x)}$ является нормированным пространством. В качестве следствия докажите, что $(C[a; b], \rho)$, $\rho(f, g) \vcentcolon = \max\limits_{x \in [a; b]}\abs{f(x) - g(x)}$ --- метрическое пространство.
\end{problem}

\begin{remark}
    В задаче $39^\circ$ у Михаила Геннадьевича, видимо, была ошибка в условии. Исправил, как сам понял. Рекомендую ознакомиться с конспектом Михаила Геннадьевича и посмотреть на формулировку в нём.
\end{remark}

\begin{solution}
    Проверим все аксиомы метрики. Первые две, очевидно, выполнены. Осталось проверить неравенство треугольника. Рассмотрим функции $f, g \in C[a; b]$, и пусть $\max\limits_{x \in [a; b]}\abs{f(x)}$ достигается при $x = x_f$, $\max\limits_{x \in [a; b]}\abs{g(x)}$ достигается при $x = x_g$, а $\max\limits_{x \in [a; b]}\abs{f(x) + g(x)}$ достигается при $x = x_{fg}$. Теперь:
    \[
        \norm{f}_1 + \norm{g}_1 = \abs{f(x_f)} + \abs{g(x_g)} \geqslant \abs{f(x_{fg})} + \abs{f(x_{fg})} \geqslant \abs{f(x_{fg}) + g(x_{fg})} = \norm{f + g}_1.
    \]

    Таким образом, выполнено неравенство треугольника для нормы. Теперь заменим, что $\rho(f, g) = \norm{f - g}_1$. Так что $\rho$ является метрикой, индуцированной нормой $\norm{\bs{\cdot}}_1$.
\end{solution}

Делать задачи 40 "---42, честно говоря, уже лениво. Они очень похожи на уже решённые.

\begin{problem}[44$^\circ$]
    Постройте пример открытых множеств в метрическом пространстве, пересечение которых не является открытым.
\end{problem}

\begin{solution}
    Можно взять любую бесконечную вложенную последовательность открытых шаров в стандартной метрике $\R^n$. Их пересечением является единственная точка, т.\,е. замкнутое множество.
\end{solution}

