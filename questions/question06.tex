\section{Теорема Дарбу. Критерий Дарбу интегрируемости по Риману}

\begin{theorem}[Дарбу]
    Пусть функция $f: [a; b] \to \R$ ограничена. Тогда следующие условия эквивалентны:
    \begin{enumerate}[nolistsep]
        \begin{minipage}{.5\textwidth}
            \item Существует $\ds(R)\int\limits_a^bf(x)dx = I$.
        \end{minipage}
        \begin{minipage}{.5\textwidth}
            \item Существует $\ds(D)\int\limits_a^bf(x)dx = I$.
        \end{minipage}
        \item Для всякого $\varepsilon > 0$ найдётся разбиение $\widetilde{T}$ отрезка $[a; b]$ такое, что $\omega(f, \widetilde{T}) < \varepsilon$.
    \end{enumerate}
\end{theorem}

\begin{proof}
    $(1) \Rightarrow (2)$. Допустим, существует $\ds (R)\int\limits_a^bf(x)dx = I$, т.\,е. для любого $\varepsilon > 0$ найдётся $\delta > 0$ такое, что соотношение
    \[
        I - \varepsilon < \mathcal{S}(f, T\xi) < I + \varepsilon
    \]
    выполнено для каждого $\delta$-разбиения и набора $\xi$ меток к нему. Тогда
    \begin{gather*}
        I - \varepsilon \leqslant \inf\limits_\xi\mathcal{S}(f, T\xi) = s(f, T) \leqslant S(f, T) = \sup\limits_\xi\mathcal{S}(f, T\xi) \leqslant I + \varepsilon,\\
        I - \varepsilon \leqslant s(f, T) \leqslant S(f, T) \leqslant I + \varepsilon;\quad
        I - \varepsilon \leqslant (D)\lowint_a^bf(x)dx \leqslant (D)\upint_a^bf(x)dx \leqslant I + \varepsilon.
    \end{gather*}
    Т.\,к. $\varepsilon > 0$ выбиралось произвольно, то $\ds I = (D)\lowint_a^bf(x)dx = (D)\upint_a^bf(x)dx = (D)\int_a^bf(x)dx$.

    $(2) \Rightarrow (3)$. Допустим, то существует $\ds (D)\int_a^bf(x)dx = I$, т.\,е. $\ds (D)\lowint_a^bf(x)dx = (D)\upint_a^bf(x)dx$. Тогда для любого $\varepsilon > 0$ найдутся разбиения $T_1$ и $T_2$ такие, что
    \[
        S(f, T_2) - \frac{\varepsilon}{2} < I < s(f, T_1) + \frac{\varepsilon}{2}.
    \]
    Если $\widetilde{T} \vcentcolon = T_1 \cap T_2$, то $\widetilde{T} \leqslant T_1, T_2$ и (по лемме 2)
    \[
        \omega(f, \widetilde{T}) = S(f, \widetilde{T}) - s(f, \widetilde{T}) \leqslant S(f, T_2) - s(f, T_1) < \varepsilon.
    \]

    $(3) \Rightarrow (1)$. Т.\,к. функция $f: [a; b] \to \R$ ограничена, найдётся $M > 0$ такое, что $\abs{f(x)} \leqslant M$ для всех $x \in [a; b]$. Выберем любое $\varepsilon > 0$ и найдём разбиение $\widetilde{T} = \{\sqbr{a_{i - 1}, a_i}\}_{i = 1}^m$ отрезка $[a; b]$ такое, что $\omega(f, \widetilde{T}) < \varepsilon$ . Положим $\delta \vcentcolon = \varepsilon / m$ и рассмотрим все концы, исключая крайние, отрезков из $\widetilde{T}$, т.\,е. точки $a_1, \ldots, a_{m - 1}$. Окружим каждую из них $\delta$-окрестностью и $2\delta$-окрестностью и возьмём объединения
    \[
        A \vcentcolon = \bigcup_{i = 1}^{m - 1}(a_i - \delta, a_i + \delta),\quad
        B \vcentcolon = \bigcup_{i = 1}^{m - 1}(a_i - 2\delta, a_i + 2\delta),\quad
    \]

    Пусть $T\xi = \{(\Delta_j, \xi_j)\}$ --- произвольное отмеченное $\delta$-разбиение отрезка $[a; b]$. Если $\xi_j \in A$, то $\Delta_j \subset B$. Если же $\xi_j \in [a; b] \setminus A$, то $\Delta_j \subset \sqbr{a_{i - 1}, a_i}$ для некоторого $i$. Имеем:
    \begin{multline*}
        \omega(f, T) = \sum_{j}\omega(f, \Delta_j)\abs{\Delta_j} = \sum_{\xi_j \in A}\omega(f, \Delta_j)\abs{\Delta_j} + \sum_{\xi_j \notin A}\omega(f, \Delta_j)\abs{\Delta_j} \leqslant\\ \leqslant 2M \cdot 4\delta m + \sum_{i = 1}^m\sum_{\Delta_j \subset \sqbr{a_{i - 1}, a_i}}\omega(f, \Delta_j)\abs{\Delta_j} \leqslant 8M\varepsilon + \sum_{i = 1}^m\omega(f, \sqbr{a_{i - 1}, a_i})\sum_{\Delta_j \subset \sqbr{a_{i - 1}, a_i}}\abs{\Delta_j} \leqslant\\ \leqslant 8M\varepsilon + \sum_{i = 1}^m \omega(f, \sqbr{a_{i - 1}, a_i})(a_i - a_{i - 1}) \leqslant 8M\varepsilon + \omega(f, \widetilde{T}) < \varepsilon C,\quad C \vcentcolon = 8M + 1.
    \end{multline*}

    Таким образом, $\omega(f, T) < \varepsilon C$. Положим $\ds I \vcentcolon = (D)\lowint_a^bf(x)dx$ (можно взять и верхний). Имеем:
    \[
        s(f, T) \leqslant \mathcal{S}(f, T\xi) \leqslant S(f, T),\quad s(f, T) \leqslant I \leqslant (D)\upint_a^bf(x)dx \leqslant S(f, T).
    \]
    Отсюда $\abs{\mathcal{S}(f, T\xi) - I} < S(f, T) - s(f, T) = \omega(f, T) < \varepsilon C$. Т.\,к. $\varepsilon > 0$ и отмеченное $\delta$-разбиение $T\xi$ произвольные, а $C > 0$ от них не зависит, то $f \in R[a; b]$ и $\ds (R)\int\limits_a^bf(x)dx = I$.
\end{proof}

\begin{corollary}[Критерий Дарбу интегрируемости по Риману]
    Если функция $f: [a; b] \to \R$ ограничена, то $\ds\exists (R)\int\limits_a^bf(x)dx = I \Leftrightarrow \exists (D)\int\limits_a^bf(x)dx = I$.
\end{corollary}

\begin{statement}
    Функция Дирихле
    \[
        \operatorname{Dir}(x) =
        \begin{cases}
            1,&\text{если $x \in \Q$},\\
            0,&\text{если $x \notin \Q$}
        \end{cases}
    \]
    не интегрируема по Риману ни на каком отрезке $[a; b]$.
\end{statement}
\begin{proof}
    В самом деле, возьмём произвольное разбиение $T = \{\Delta_i\}$ отрезка $[a; b]$. В каждом отрезке $\Delta_i$ есть точки как из $\Q$, так и не из $\Q$. Следовательно,
    \[
        s(\operatorname{Dir}, T) = \sum_i\inf\limits_{\Delta_i}\operatorname{Dir} \cdot \abs{\Delta_i} = 0,\quad S(\operatorname{Dir}, T) = \sum_i\sup\limits_{\Delta_i}\operatorname{Dir} \cdot \abs{\Delta_i} = \sum_i 1 \cdot \abs{\Delta_i} = b - a.
    \]
    Значит, $\ds(D)\lowint\limits_a^b\operatorname{Dir} = 0$ и $\ds (D)\upint\limits_a^b\operatorname{Dir} = b - a$. Несовпадение интегралов даёт $f \notin R[a; b]$.
\end{proof}

\begin{problem}
    Докажите, что функция Римана
    \[
        \operatorname{Riem}(x) =
        \begin{cases}
            \frac{1}{n},&\text{если $x = \frac{m}{n}$, $\frac{m}{n}$ --- несократимая дробь},\\
            0,&\text{если $x \notin \Q$}
        \end{cases}
    \]
    интегрируема по Риману на каждом отрезке $[a; b]$ и вычислите $\ds (R)\int\limits_a^b\operatorname{Riem}(x)dx$.
\end{problem}

\begin{solution}
    Пока не решил, но знаю правильный ответ --- интеграл функции Римана равен $0$ на любом отрезке.
\end{solution}

