\section{Интегральные суммы Римана. Интеграл Римана. Интегрируемость по Риману и ограниченность}

\begin{definition}
    Возьмём отрезок $[a; b]$ и построим конечный набор точек
    \[
        a = a_0 < a_1 < \ldots < a_{m - 1} < a_m = b,
    \]
    тем самым разбив $[a; b]$ на попарно неперекрывающиеся отрезки $\Delta_i \vcentcolon = [a_{i - 1}; a_i]$, $1 \leqslant i \leqslant m$. Набор $T = \{\Delta_i\}_{i = 1}^m$ этих отрезков --- \textit{разбиение} отрезка $[a; b]$.
\end{definition}

\begin{definition}
    Добавим к $T$ произвольный набор $\xi = \{\xi_i\}_{i = 1}^m$ точек $\xi_i \in \Delta_i$ (\textit{меток разбиения} $T$). \textit{Отмеченное разбиение} $T\xi$ отрезка $[a; b]$ --- это множество пар $\{(\Delta_i, \xi_i)\}_{i = 1}^m$.
\end{definition}

\begin{definition}
    \textit{Интегральная сумма} (\textit{сумма Римана}) функции $f: [a; b] \to \R$, соответствующая отмеченному разбиению $T\xi = \{(\Delta_i, \xi_i)\}_{i = 1}^m$ отрезка $[a; b]$, есть сумма
    \[
        \mathcal{S}(f, T\xi) \vcentcolon = \sum_{i = 1}^mf(\xi_i)|\Delta_i|.
    \]
\end{definition}

\textbf{Геометрический смысл сумм Римана}. Рассмотрим определённую на $[a; b]$ неотрицательную функцию $f$ и криволинейную трапецию
\[
    A = A_{f, [a; b]} \vcentcolon = \left\{(x, y) \in \R^2\;\middle\vert\; x \in [a; b] \wedge y \in [0; f(x)]\right\} \subset \R^2,
\]
связанную с графиком функции $f$. Тогда интегральная сумма $\mathcal{S}(f, T\xi)$ совпадает с площадью объединения прямоугольников, построенных на отрезка разбиения $T$ как на основаниях и имеющих высоту $f(\xi_i)$:

\begin{center}
    \begin{asy}
        size(10cm);
        import graph;
        draw((0, -0.7)--(0, 2), Arrow(HookHead, 1.5mm));
        draw((-0.5, 0)--(5.5, 0), Arrow(HookHead, 1.5mm));

        real f(real x)
        {
            return 0.2 * (-1/3 * x * x * x + x * x + 2 * x + 3);
        }

        real x;
        for (x = 0.5; x < 4.5; x += 0.5)
        {
            fill((x, 0)--(x + 0.5, 0)--(x + 0.5, f(x + 0.3))--(x, f(x + 0.3))--cycle, palecyan);
            draw((x, 0)--(x, f(x + 0.3))--(x + 0.5, f(x + 0.3))--(x + 0.5, 0));
            draw((x + 0.3, 0)--(x + 0.3, f(x + 0.3)), dashed);
            dot("${}$", (x + 0.3, f(x + 0.3)), dir(0));
            /* % */ label(format("$\xi_%d$", (int)(2 * x)), (x + 0.3, -0.2));
        }

        draw(graph(f, -0.3, 5), currentpen + 1);
    \end{asy}
\end{center}

\begin{definition}
    \textit{Диаметр} разбиения $T = \{\Delta_i\}_{i = 1}^m$ --- число
    \[
        d(T) \vcentcolon = \underset{i = 1, \ldots, m}{\max}|\Delta_i|.
    \]
\end{definition}

\begin{definition}
    Если $d(T) < \delta$, то разбиение $T$ назовём \textit{$\delta$-разбиением}.
\end{definition}

\begin{definition}[Интеграл Римана]
    Функция $f: [a; b] \to \R$ называется \textit{интегрируемой по Риману} на отрезке $[a; b]$ (пишем $f \in R[a; b]$), если существует число $I \in \R$ такое, что для любого $\varepsilon > 0$ найдётся $\delta > 0$ такое, что для любого $\delta$-разбиения $T\xi = \{(\Delta_i, \xi_i)\}_{i = 1}^m$ отрезка $[a; b]$ выполнено неравенство
    \[
        \abs{\mathcal{S}(f, T\xi) - I} = \abs{\sum_{i = 1}^mf(\xi)|\Delta_i| - I} < \varepsilon.
    \]
    Число $I$ называют \textit{интегралом Римана} функции $f$ по отрезку $[a; b]$, обозначается $\ds\int\limits_a^bf(x)dx$.
\end{definition}

Следующая задача даёт способ вычисления интеграла Римана:

\begin{problem}
    (А) Пусть $f \in R[a; b]$, $\{T^n\xi^n\}_{n = 1}^\infty$ --- последовательность отмеченных разбиений отрезка $[a; b]$, причём $d(T^n) \to 0$ при $n \to \infty$. Докажите, что
    \[
        \int\limits_a^bf(x)dx = \lim_{n \to \infty}\mathcal{S}(f, T^n\xi^n).
    \]

    (Б) Разобъём отрезок $[a; b]$ на $n$ равных отрезков $\Delta_j^{(n)}$, $1 \leqslant j \leqslant n$ длины $(b - a) / n$, а затем в каждом из них произвольным образом выберем метку $\xi_j^{(n)}$. Покажите, что
    \[
        \int\limits_a^bf(x)dx = \lim_{n \to \infty}\frac{b - a}{n}\sum_{j = 1}^nf(\xi_j^{(n)}).
    \]
\end{problem}

\begin{solution}
    (А) По определению предела последовательности $d(T^n)$,
    \[
        \forall \varepsilon > 0\;\exists \mathcal{N}_\varepsilon: (n > \mathcal{N}_\varepsilon) \Rightarrow \br{\abs{d(T^n)} < \varepsilon}\eqno(\ast)
    \]
    По определению интеграла Римана
    \[
        \forall \varepsilon > 0\; \exists \delta > 0: \br{d(T) < \delta} \Rightarrow \br{\abs{\mathcal{S}(f, T\xi) - I} < \varepsilon}.
    \]
    Перепишем последнее высказывание с учётом $(\ast)$:
    \[
        \forall\varepsilon > 0\; \exists N_\varepsilon \vcentcolon = \mathcal{N}_\varepsilon: (n > N_\varepsilon) \Rightarrow (\abs{\mathcal{S}(f, T\xi) - I} < \varepsilon).
    \]
    Значит, утверждение теоремы верно по опредлению предела последовательности.

    (Б) По предыдущему пункту
    \[
        \int\limits_a^bf(x)dx = \lim_{n \to \infty}\mathcal{S}(f, T^n\xi^n) = \lim_{n \to \infty}\br{\sum_{j = 1}^nf(\xi_j^{(n)})|\Delta_j^{(n)}|} = \lim_{n \to \infty}\frac{b - a}{n}\sum_{j = 1}^nf(\xi_j^{(n)}).
    \]
\end{solution}

\begin{proposal}
    Если $f$ интегрируема по Риману на отрезке $[a; b]$, то она ограничена на $[a; b]$.
\end{proposal}

\begin{proof}
    Допустим, $f \in R[a; b]$, но $f \notin B[a; b]$. Возьмём любые $C > 0$ и разбиение $T = \{\Delta_i\}$. Тогда $f$ не ограничена на $\Delta_i$ по крайней мере для одного $i$ ($=\vcentcolon i_0$). При всех $i \ne i_0$ расставим метки $\xi_i \in \Delta_i$ произвольным образом, а метку $\xi_{i_0} \in \Delta_{i_0}$ выберем так, что
    \[
        \abs{\mathcal{S}(f, T\xi)} = \abs{\sum_if(\xi_i)\abs{\Delta_i}} \geqslant \abs{f(\xi_{i_0})\abs{\Delta_{i_0}}} - \abs{\sum_{i \ne i_0}f(\xi_i)\abs{\Delta_i}} > C.
    \]

    Это возможно благодаря неограниченности $f$ на отрезке $\Delta_{i_0}$. Итак, для всех разбиений $T$ имеем $\sup\limits_\xi\abs{\mathcal{S}(f, T\xi)} = \infty$, поэтому ни для какого $\varepsilon > 0$ мы не сможем найти $\delta > 0$ такое, что неравенство \[\ds\abs{\mathcal{S}(f, T\xi) - \int\limits_a^bf(x)dx} < \varepsilon\] выполнялось бы для всех отмеченных $\delta$-разбиений $T\xi$, что противоречит тому, что $f \in R[a; b]$.
\end{proof}

