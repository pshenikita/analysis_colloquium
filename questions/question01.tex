\section{Первообразная, обобщённая первообразная, неопределённый интеграл. Теоремы о множестве всех первообразных и обобщённых первообразных. Дифференцирование и интегрирование --- обратные операции. Интегрирование --- линейная операция}

\begin{definition}
    Пусть функция $f$ определена на промежутке $X$. \textcolor{gray}{(Непрерывная)} функция $F$ на $X$ называется \textit{первообразной} \textcolor{gray}{(\textit{обобщённой первообразной})} функции $f$, если $F^\prime(x) = f(x)$ для всех $x \in X$ \textcolor{gray}{(для всех $x \in X$, кроме конечного числа)}.
\end{definition}

\begin{remark}
    Если функция $f$ непрерывна на промежутке $X$, то на этом промежутке для неё существует первообразная. Если $f$ кусочно-непрерывна на промежутке $X$, то на $X$ для неё существует обобщённая первообразная. Доказано это будет позднее.
\end{remark}

\begin{statement}
    $F^\prime \equiv 0$ на $X \iff F = const$.
\end{statement}

\begin{proof}
    $\Leftarrow$ Очевидно. $\Rightarrow$ По теореме Лагранжа \[\forall x_1, x_2 \in X\ F(x_1) - F(x_2) = \underbrace{F^\prime(c)}_{{} = 0}(x_1 - x_2) = 0\] для некоторого $c \in X$, значит, $F = const$.
\end{proof}

Аналогичное утверждение верно и для обобщённой первообразной, достаточно провести вышеописанное доказательство для отрезков между выкинутыми точками. Оно всё ещё корректно, т.\,к. в теореме Лагранжа требуется дифференцируемость только во внутренних точках отрезка.

\begin{theorem}[О множестве \textcolor{gray}{(обобщённых)} первообразных]
    Если $F_1$ и $F_2$ --- \textcolor{gray}{(обобщённые)} первообразные $f$ на $X$, то $F_1 - F_2 = const$.
\end{theorem}

\begin{proof}
    $(F_1 - F_2)^\prime = f - f = 0$.
\end{proof}

Произвольная первообразная функции $f$ на промежутке $X$ обозначается через $\ds\int f(x) dx$. Если $F$ --- первообразная $f$, то пишут $\ds\int f(x)dx = F(x) + C$. Первообразную $\ds\int f(x)dx$ называют \textit{неопределённым интегралом}.

Нетрудно заметить, что выполняется следующее:
\[
    \int dF = \int F^\prime(x)dx = F(x) + C,\qquad d\br{\int f(x)dx} = f(x)dx.
\]

Поэтому говорят, что дифференцирвание и интегрирование --- обратные операции. Известно, что дифференцирование --- линейная операция, т.\,е.
\[
    d(\alpha F + \beta G) = \alpha \cdot dF + \beta \cdot dG.
\]

Возьмём первообразную обеих частей, получим
\[
    \int\br{\alpha dF + \beta dG} = \int\br{\alpha f + \beta g}dx = \alpha\int f(x)dx + \beta\int g(x)dx.
\]

Поэтому говорят, что интегрирование --- линейная операция.

