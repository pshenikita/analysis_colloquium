\section{Вычисление первообразных непосредственным интегрированием, интегрированием по частям и заменой переменной. Примеры}

Указанные ниже равенства верны на соответствующих областях определения (промежутках):

\begin{enumerate}
    \begin{minipage}{.33\textwidth}
        \item $\ds\int 0dx = const$.
        \item $\ds\int x^\alpha dx = \frac{x^{\alpha + 1}}{\alpha + 1} + C, \alpha \ne -1$
        \item $\ds\int\frac{dx}{x} = \ln\abs{x} + C$
    \end{minipage}
    \begin{minipage}{.33\textwidth}
        \item $\ds\int e^xdx = e^x + C$
        \item $\ds\int\cos xdx = \sin x + C$
        
            $\ds\int\sin xdx = -\cos x + C$
    \end{minipage}
    \begin{minipage}{.33\textwidth}
        \item $\ds\int \frac{dx}{\cos^2x} = \tg x + C$

            $\ds\int \frac{dx}{\sin^2x} = -\ctg x + C$
        \item $\ds\int\ddfrac{dx}{\sqrt{1 - x^2}} = \arcsin x + C$

            $\ds\int\ddfrac{dx}{1 + x^2} = \arctg x + C$
    \end{minipage}
    \item Длинный логарифм:
        \[
            \br{\ln\abs{x + \sqrt{x^2 \pm 1}}}^\prime = \ddfrac{1 + \frac{2x}{2\sqrt{x^2 \pm 1}}}{x + \sqrt{x^2 \pm 1}} = \frac{1}{\sqrt{x^2 \pm 1}},\quad \int\ddfrac{dx}{\sqrt{x^2 \pm 1}} = \ln\abs{x + \sqrt{x^2 + 1}} + C
        \]

        Высокий логарифм:
        \begin{multline*}
            \int\frac{dx}{x^2 - 1} = \int\frac{dx}{(x - 1)(x + 1)} = \int\frac{1}{2}\br{\frac{1}{x - 1} - \frac{1}{x + 1}}dx =\\ = \frac{1}{2}\int\frac{dx}{x - 1} + \frac{1}{2}\int\frac{dx}{x + 1} = \frac{1}{2}\ln\abs{x - 1} - \frac{1}{2}\ln\abs{x + 1} + C = \frac{1}{2}\ln\abs{\frac{x - 1}{x + 1}} + C
        \end{multline*}
\end{enumerate}

По правилу Лейбница,
\[
    d(uv) = du \cdot v + u \cdot dv.
\]

Найдём первообразную от обеих частей:
\[
    \int vdu + \int udv = uv + C\quad\text{или же}\quad\int u^\prime vdx = uv - \int u v^\prime dx.
\]

Эта формула называется \textit{формулой интегрирования по частям}.

По правилу дифференцированию сложной функции
\[
    (F(\varphi(t)))^\prime = F^\prime(\varphi(t))\varphi^\prime(t).
\]

Найдём первообразную от обеих частей:
\[
    \int f(\varphi(t))\varphi^\prime(t)dt = \left.\int f(x)dx\right\vert_{x = \varphi(t)}.
\]

Читать последнее равенство также можно как $\ds\int f(\varphi(t))\underbrace{\varphi^\prime(t)dt}_{d\varphi} = \int f(\varphi)d\varphi$.

