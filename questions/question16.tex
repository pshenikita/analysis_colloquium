\section{Вариация функции и функции ограниченной вариации ($VB$-функции). О связи ограниченности вариации с монотонностью и ограниченностью функции. Аддитивность вариации и структура $VB$-функции}

\begin{definition}
    \textit{Вариация} функции $f: [a; b] \to \R$ (на отрезке $[a; b]$) --- величина
    \[
        \V_a^bf \vcentcolon = \sup\limits_T\sum_{i = 1}^m\abs{f(a_i) - f(a_{i - 1})},
    \]
    где $\sup$ берётся по всем разбиениям $T = \{[a_{i - 1}, a_i]\}_{i = 1}^m$ отрезка $[a; b]$. Если $\V\limits_a^bf < +\infty$, то $f$ --- \textit{функция ограниченной вариации} на $[a; b]$. Запись: $f \in BV[a; b]$.
\end{definition}

Для разбиения $T = \{[a_{i - 1}, a_i]\}_{i = 1}^m$ отрезка $[a; b]$ введём обозначение
\[
    V(f, T) \vcentcolon = \sum_{i = 1}^m\abs{f(a_i) - f(a_{i - 1})}.
\]

\begin{proposal}
    Любая монотонная на отрезке функция имеет ограниченную вариацию. При этом $\V\limits_a^bf$ равна $f(b) - f(a)$, если $f$ не убывает и $f(a) - f(b)$, если $f$ не возрастает.
\end{proposal}

\begin{proof}
    Пусть $T = \{[a_{i - 1}, a_i]\}_{i = 1}^m$ --- произвольное разбиение отрезка $[a; b]$. Если $f$ не убывает, то
    \begin{multline*}
        V(f, T) = \sum_{i = 1}^m\abs{f(a_i) - f(a_{i - 1})} = \sum_{i = 1}^m(f(a_i) - f(a_{i - 1})) = f(b) - f(a),\\ \V_a^bf = \sup\limits_T V(f, T) = f(b) - f(a).
    \end{multline*}

    Другой случай рассматривается аналогично.
\end{proof}

\begin{proposal}
    Любая функция ограниченной вариации ограничена.
\end{proposal}

\begin{proof}
    Для каждого $x \in [a; b]$ имеем
    \[
        2\abs{f(x)} \leqslant \abs{f(x) - f(a)} + \abs{f(b) - f(x)} + \abs{f(a)} + \abs{f(b)}.
    \]
    Набор из двух отрезков $[a; x]$ и $[x; b]$ есть разбиение $[a; b]$, поэтому
    \[
        \abs{f(x) - f(a)} + \abs{f(b) - f(x)} \leqslant \V_a^bf.
    \]
    В итоге,
    \[
        \abs{f(x)} \leqslant \frac{1}{2}\br{\V_a^bf + \abs{f(a)} + \abs{f(b)}}\text{ для всех $x \in [a; b]$}.
    \]
    Отсюда следует требуемое.
\end{proof}

\begin{theorem}[Об аддитивности вариации]
    Если $f \in BV[a; b]$ и $a < c < b$, то $\V\limits_a^bf = \V\limits_a^cf + \V\limits_c^bf$.
\end{theorem}

\begin{proof}
    Возьмём произвольное $\varepsilon > 0$. Найдём разбиения $T_1$ и $T_2$ отрезков $[a; c]$ и $[c; b]$, соотвественное, такие, что
    \[
        V(f, T_1) > \V_a^cf - \varepsilon\quad\text{и}\quad V(f, T_2) > \V_c^bf - \varepsilon.
    \]
    Тогда $T_1 \sqcup T_2$ --- разбиение отрезка $[a; b]$ и
    \[
        \V_a^bf \geqslant V(f, T_1 \sqcup T_2) = V(f, T_1) + V(f, T_2) > \V_a^cf + \V_c^bf - 2\varepsilon.
    \]
    Т.\,к. $\varepsilon > 0$ выбиралось произвольно,
    \[
        \V_a^bf \geqslant \V_a^cf + \V_c^bf.
    \]
    Докажем обратное неравенство. Для каждого $\varepsilon > 0$ отыщем разбиение $T$ отрезка $[a; b]$, для которого
    \[
        V(f, T) > \V_a^bf - \varepsilon.
    \]

    Если порождающий разбиение $T$ набор точек не содержит $c$, добавим её в этот набор и получим новое разбиение (оставим ему старое обозначение $T$), для которого тем более выполнено последнее. При этом $T = T_1 \sqcup T_2$ --- разбиения отрезков $[a; c]$ и $[c; b]$. Получаем
    \[
        \V_a^cf + \V_c^bf \geqslant V(f, T_1) + V(f, T_2) = V(f, T) > \V_a^bf - \varepsilon,
    \]
    откуда
    \[
        \V_a^bf \leqslant \V_a^cf + \V_c^bf.
    \]
\end{proof}

\begin{lemma}
    Если $f \in BV[a; b]$, то:
    \begin{enumerate}[nolistsep]
        \item функции $V$ и $V - f$ не убывают на $[a; b]$, $V(x) \vcentcolon = \V\limits_a^xf$;
        \item $f$ есть разность двух неубывающих функций.
    \end{enumerate}
\end{lemma}

\begin{proof}
    \begin{enumerate}
        \item Пусть $x_2 > x_1$. Имеем:
            \[
                V(x_2) = \V\limits_a^{x_2}f = \V\limits_a^{x_1} + \V\limits_{x_1}^{x_2} \geqslant \V\limits_a^{x_1}f = V(x_1);\quad V(x_2) - V(x_1) = \V\limits_{x_1}^{x_2} \geqslant \abs{f(x_2) - f(x_1)} \geqslant f(x_2) - f(x_1),
            \]
            а значит, $(V - f)(x_2) \geqslant (V - f)(x_1)$.
        \item $f$ есть разность функций $V$ и $V - f$, которые не убывают, согласно п.\,1.
    \end{enumerate}
\end{proof}

\begin{theorem}[О структуре $BV$-функций]
    $f \in BV[a; b] \Leftrightarrow f$ есть разность двух неубывающих функций.
\end{theorem}

\begin{proof}
    $\Rightarrow$. Доказано в предыдущей лемме.

    $\Leftarrow$. Пусть $f = g - h$, где $g$ и $h$ не убывают на $[a; b]$. Возьмём произвольное разбиение $T = \{[a_{i - 1}; a_i]\}_{i = 1}^m$ отрезка $[a; b]$. Тогда:
    \begin{multline*}
        V(f, T) = \sum_{i = 1}^m\abs{g(a_i) + h(a_i) - g(a_{i - 1}) - g(a_{i - 1})} \leqslant \sum_{i = 1}^m\abs{g(a_i) - g(a_{i - 1})} + {}\\{} + \sum_{i = 1}^m\abs{h(a_i) - h(a_{i - 1})} = g(b) - g(a) + h(b) - h(a);
    \end{multline*}
    \[
        \V\limits_a^bf = \sup\limits_TV(f, T) \leqslant g(b) - g(a) + h(b) - h(a) < \infty.
    \]
    Значит, $f \in BV[a; b]$ по определению.
\end{proof}

