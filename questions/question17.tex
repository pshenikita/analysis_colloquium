\section{Вариация непрерывно дифференцируемых функций. Спрямляемые кривые, критерий спрямляемости}

\begin{theorem}
    Допустим, $f \in C^{(1)}[a; b]$. Тогда $f \in BV[a; b]$ и $\ds\V_a^bf = \int\limits_a^b\abs{f^\prime(x)}dx$.
\end{theorem}

\begin{proof}
    По условию, $f^\prime \in C[a; b]$, значит, $\abs{f^\prime} \in R[a; b]$ и $\ds\int\limits_a^b\abs{f^\prime(x)}dx = \vcentcolon I$ определён. Докажем, что $\V\limits_a^bf = I$.

    Сначала установим неравенство $\V\limits_a^bf < +\infty$. Возьмём любое разбиение $T = \{[a_{i - 1}; a_i]\}_{i = 1}^m$ отрезка $[a; b]$ и оценим $V(f, T)$:
    \begin{multline*}
        V(f, T) = \sum_{i = 1}^m\abs{f(a_i) - f(a_{i - 1})} \overset{\text{т.\,Лагранжа}}{=\joinrel=} \sum_{i = 1}^m\abs{f^\prime(\xi_i)}(a_i - a_{i - 1}) \leqslant\\\leqslant \max\limits_{x \in [a; b]}\abs{f^\prime(x)}\sum_{i = 1}^m(a_i - a_{i - 1}) = \max\limits_{x \in [a; b]}\abs{f^\prime(x)}(b - a).
    \end{multline*}

    Мы воспользовались тем, что функция $\abs{f^\prime}$ непрерывна, а потому ограничена на $[a; b]$. Из последнего неравенства мы видим, что $\V\limits_a^b \leqslant \max\limits_{x \in [a; b]}\abs{f^\prime(x)}(b - a) < +\infty$.

    Далее, возьмём произвольное $\varepsilon > 0$. Найдётся $\delta > 0$ такое, что для всякого отмеченного $\delta$-разбиения $T\xi$ отрезка $[a; b]$ верно $\abs{\mathcal{S}(\abs{f^\prime}, T\xi) - I} < \varepsilon$. Найдём разбиение $T = \{[a_{i - 1}; a_i]\}_{i = 1}^m$, для которого
    \[
        \V\limits_a^bf - \varepsilon < V(f, T) \leqslant \V_a^bf.
    \]

    Размельчая $T$ так, чтобы диаметр стал меньше $\delta$, мы не уменьшим $V(f, T)$ и сохраним последнее неравенство. Поэтому сразу считаем $d(T) < \delta$. Получаем: $V(f, T) = \mathcal{S}(\abs{f^\prime}, T\xi)$.
    \[
        \abs{\V_a^bf - I} \leqslant \abs{\V_a^bf - V(f, T)} + \abs{V(f, T) - \mathcal{S}(\abs{f^\prime}, T\xi)} + \abs{\mathcal{S}(\abs{f^\prime}, T\xi) - I} \leqslant 2\varepsilon.
    \]

    Т.\,к. $\varepsilon > 0$ выбиралось произвольным, $\V\limits_a^bf = I$.
\end{proof}

\begin{definition}
    \textit{Плоская кривая} $\gamma \subset \R^2$ задаётся параметрически заданной функцией (\textit{путём})
    \[
        \begin{cases}
            x = x(t),\\
            y = y(t)
        \end{cases}\quad t \in [a; b].
    \]
    Формально, $\gamma$ есть образ отрезка $[a; b]$ при отображении
    \[
        t \in [a; b] \mapsto (x(t), y(t)) \in \R^2.
    \]
\end{definition}

\begin{remark}
    Считаем $x(t), y(t) \in C[a; b]$; в этом случае как сама кривая, так и задающий её путь называется \textit{непрерывными}. Также предполагаем, что $\gamma$ --- простая кривая без кратных точек, что означает следующее. Если $P_1 = (x(t_1), y(t_1))$ и $P_2 = (x(t_2), y(t_2))$ и $t_1 \ne t_2$, то $P_1 \ne P_2$.
\end{remark}

Разобъём кривую точками $P_i$ ($i = 0, \ldots, m$) на $m$ дуг. Т.\,к. кривая не имеет самопересечений, такому разбиению на дуги однозначно соответствует некоторое разбиение $T = \{\Delta_i = [a_{i - 1}; a_i]\}_{i = 1}^m$ отрезка $[a; b]$. А именно, если $(x_i, y_i)$ --- координаты точек $P_i$, то $x_i = x(a_i)$ и $y_i - y(a_i)$ для $i = 0, \ldots, m$.

Длина хорды, стягивающей дугу $P_{i - 1}P_i$ есть $\abs{P_{i - 1}P_i}$. Длина $\ell(P_0P_1\ldots P_m)$ всей ломаной равна $\sum\limits_{i = 1}^m\abs{P_{i - 1}P_i}$.

\begin{definition}
    Если $\ell(\gamma) < +\infty$,
    \[
        \ell(\gamma) \vcentcolon = \sup_{P_0P_1\ldots P_m}\sum_{i = 1}^m\abs{P_{i - 1}P_i} = \sup_T\sum_{i = 1}^m\abs{P_{i - 1}P_i},
    \]
    то кривая $\gamma$ называется \textit{спрямляемой}, а $\ell(\gamma)$ --- её \textit{длиной}.
\end{definition}

\begin{theorem}[Критерий спрямляемости кривой]
    Плоская непрерывная кривая
    \[
        \gamma:
        \begin{cases}
            x = x(t),\\
            y = y(t)
        \end{cases}\quad t \in [a; b]
    \]
    без кратных точек спрямляема тогда и только тогда, когда $x(t)$ и $y(t)$ --- функции ограниченной вариации.
\end{theorem}

\begin{proof}
    $\Rightarrow$. Допустим, $\gamma$ --- спрямляемая кривая, т.\,е.
    \[
        \sup_T\sum_{i = 1}^m\abs{P_{i - 1}P_i} = \ell(\gamma) < +\infty.
    \]

    Тогда имеем
    \[
        \sum_{i = 1}^m\abs{x(a_i) - x(a_{i - 1})} \leqslant \sum_{i = 1}^m\abs{P_{i - 1}P_i} \leqslant \sup_T\sum_{i = 1}^m\abs{P_{i - 1}P_i} = \ell(\gamma).
    \]

    Значит, $\V\limits_a^bx(t) \leqslant \ell(\gamma) < +\infty$, т.\,е. $x(t) \in BV[a; b]$. Аналогично, $y(t) \in BV[a; b]$.

    $\Leftarrow$. Пусть $\V\limits_a^bx(t) < +\infty$ и $\V\limits_a^by(t) < +\infty$. Тогда 
    \[
        \sum_{i = 1}^m\abs{P_{i - 1}P_i} \leqslant \sum_{i = 1}^m\abs{x(a_i) - x(a_{i - 1})} + \sum_{i = 1}^m\abs{y(a_i) - y(a_{i - 1})} \leqslant \V_a^bx + \V_a^by.
    \]

    Значит, $\ell(\gamma) = \sup\limits_T\sum\limits_{i = 1}^m\abs{P_{i - 1}P_i} \leqslant \V\limits_a^b + \V\limits_a^by < +\infty$.
\end{proof}

\begin{remark}
    Все неравенства выше --- это просто следствия из неравенства треугольника.
\end{remark}

