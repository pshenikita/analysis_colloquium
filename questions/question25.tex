\section{Открытые и замкнутые множества в метрическом пространстве, критерий замкнутости. Открытость и замкнутость шаров. Пересечения и объединения открытых и замкнутых множеств. Замкнутость границы множества}

\begin{definition}
    $E$ --- \textit{открытое} множество в метрическом пространстве $(X, \rho)$, если $E = \inter E$. $E$ --- \textit{замкнутое} множество в $(X, \rho)$, если $E^\prime \subseteq E$.
\end{definition}

Из определений открытого и замкнутого множеств вытекает, что каждое из множеств $\varnothing$ и $X$ является и открытым, и замкнутым.

\begin{proposal}
    Внутренность и внешность любого множества открыты.
\end{proposal}

\begin{proof}
    Если $a$ --- внутренная точка $E$, найдётся окрестность $U(a) \subseteq E$. Но $U(a)$ является окрестностью всех лежащих в ней точек. Значит, все точки $x \in U(a)$ есть внутренние точки множества $E$, т.\,е. $U(a) \subseteq \inter E$. Следовательно, $\inter E$ --- открытое множество. Доказательство открытости $\ext E$ проходит аналогично.
\end{proof}

\begin{theorem}[Критерий замкнутости]
    Следующие условия равносильны:
    \begin{enumerate}[nolistsep]
        \item $E$ замкнуто;
        \item $X \setminus E$ открыто;
        \item $\partial E \subseteq E$.
    \end{enumerate}
\end{theorem}

\begin{proof}
    $1 \Rightarrow 2$. Допустим, $E$ замкнуто. Возьмём какую-нибудь точку $x \in X \setminus E$. Если $x$ была бы предельной точкой множества $E$, то $x \in E$, т.\,к. $E$ замкнуто, но $x \notin E$. Следовательно, $x$ не является предельной точкой для $E$. Значит, найдётся проколотая окрестность $\mathring{U}(x)$, где нет точек из $E$. Сама точка $x$ также не находится в $E$, поэтому в окрестности $U(x)$ нет точек из $E$, т.\,е. $U(x) \subseteq X \setminus E$. Итак, $\forall x \in X \setminus E$ мы нашли окрестность $U(x) \subseteq X \setminus E$. Это означает, что множество $X \setminus E$ открыто.
    
    $2 \Rightarrow 3$. Предположим, что $X \setminus E$ открыто. Возьмём произвольную граничную точку $x$ множества $E$ и допустим, что $x \notin E$. Тогда ($X \setminus E$ открыто) найдётся окрестность $U(x) \subseteq X \setminus E$. Но в этом случае $x$ не может быть граничной точкой множества $E$ --- противоречие. Значит, каждая граничная точка $x$ лежит в $E$.
    
    $3 \Rightarrow 1$. Пусть $x$ --- любая предельная для $X$ точка; необходимо показать, что $x \in E$. Точка $x$ может быть внутренней, граничной или внешней точкой множества $E$. Если $x$ внутренняя, то $x \in E$ согласно определению внутренней точки. Если $x$ граничная, то $x \in E$ по условию. Если же $x$ внешняя точка для $E$, то найдётся окрестность $U(x)$, в которой нет точек из $E$. Однако это невозможно, т.\,к. $x$ --- предельная точка множества $X$.
\end{proof}

\begin{proposal}
    Открытый шар --- открытое множество, замкнутый шар --- замкнутое множество.
\end{proposal}

\begin{proof}
    Пусть $B_r(a)$ --- открытый шар и $x \in B_r(a)$. Тогда этот шар является окрестностью точки $x$, т.\,е. он открыт.

    Пусть $\overline{B}_r(a)$ --- замкнутый шар, $x \in \overline{B}_r(a)^\prime$. Случай $\rho(x, a) \vcentcolon = r_1 > r$ невозможен, т.\,к. в этом случае $B_\delta(x) \cap \overline{B}_r(a) = \varnothing$ при $\delta = (r_1 - r) / 2$, поэтому $x \notin \overline{B}_r(a)^\prime$. Значит, $\rho(x, a) \leqslant r$, т.\,е. $x \in \overline{B}_r(a)$. Итак, $\overline{B}_r(a)^\prime \subset \overline{B}_r(a)$, т.\,е. шар $\overline{B}_r(a)$ замкнут.
\end{proof}

\begin{theorem}
    \begin{enumerate}[nolistsep]
        \item Объединение открытых множеств открыто, пересечение замкнутых --- замкнуто.
        \item Пересечение конечного числа открытых множеств открыто, объединение конечного числа замкнутых --- замкнуто.
    \end{enumerate}
\end{theorem}

\begin{proof}
    \begin{enumerate}
        \item Пусть все множества $\{U_i : i \in I\}$ (где $I$ --- некоторое множество индексов) открыты. Возьмём любую точку $x \in \bigcup\limits_{i \in I}U_i$. Тогда $x \in U_\alpha$ для некоторого $\alpha \in I$, а т.\,к. $U_\alpha$ открыто, то найдётся окрестность $V(x) \subset U_\alpha$. Но тогда $V(x) \subset \bigcup\limits_{i \in I}U_i$, откуда множество $\bigcup\limits_{i \in I}U_i$ открыто.

            Теперь пусть все множества $\{F_i : i \in I\}$ замкнуты, т.\,е. $F_i = X \setminus U_i$, где $U_i$ открыто $\forall i \in I$. Имеем
            \[
                \bigcap_{i \in I}F_i = \bigcap_{i \in I}(X \setminus U_i) = X \setminus \br{\bigcup_{i \in I}U_i}.
            \]
            При этом $\bigcup\limits_{i \in I}U_i$ открыто, поэтому его дополнение $\bigcap\limits_{i \in I}F_i$ замкнуто.
        \item Пусть $x \in \bigcap\limits_{i = 1}^nU_i$, тогда $x \in U_i$ $\forall i = 1, \ldots, n$. Если все множества $U_i$ открыты, то для каждого $k$ можно найти шар $B_{r_i}(x) \subset U_i$. Множество $V(x) \vcentcolon = \bigcap\limits_{i = 1}^nB_{r_i}(x)$ есть окрестность точки $x$, которая лежит в $\bigcap\limits_{i = 1}^nU_i$. Следовательно, множество $\bigcap\limits_{i = 1}^nU_i$ открыто.

            Допустим, для каждого $i = 1, \ldots, n$ множество $F_i$ замкнуто: $F_i = X \setminus U_i$, где $U_i$ открыто. Тогда
            \[
                \bigcup_{i = 1}^nF_i = \bigcup_{i = 1}^n(X \setminus U_i) = X \setminus \br{\bigcap_{i = 1}^nU_i}.
            \]
            При этом $\bigcap\limits_{i = 1}^nU_i$ открыто, поэтому его дополнение $\bigcup\limits_{i = 1}^nF_i$ замкнуто.
    \end{enumerate}
\end{proof}

