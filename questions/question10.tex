\section{Свойства интеграла Римана (интегрируемость изменённой функции, достаточное условие положительности интеграла, интеграл по симметричному отрезку от чётных и нечётных функций, интегрируемость кусочно-непрерывных функций)}

\begin{theorem}[Об интегрируемости изменённой функции]
    Если функцию $f \in R[a; b]$ изменить на конечном множестве, то изменённая функция $\widetilde{f} \in R[a; b]$ и $\ds\int\limits_a^bf = \int\limits_a^b\widetilde{f}$.
\end{theorem}

Нам понадобится лемма:

\begin{lemma}
    Допустим, $E \subset [a; b]$ --- конечное множество, и функция $g$ равна нулю вне $E$. Тогда $g \in R[a; b]$ и $\ds\int\limits_a^bg(x)dx = 0$.
\end{lemma}

\begin{proof}
    Пусть $E = \{a_1, \ldots, a_n\}$, $C \vcentcolon = \max\{f(a_1), \ldots, f(a_n)\}$. Выберем любое $\varepsilon > 0$, положим $\delta \vcentcolon = \varepsilon / (Cn)$ и возьмём произвольное $\delta$-разбиение $T\xi = \{(\Delta_i, \xi_i)\}$ отрезка $[a; b]$. Тогда
    \[
        \abs{\mathcal{S}(f, T\xi)} < \abs{\sum_{\xi_i \in E}g(\xi_i)\abs{\Delta_i}} + \abs{\sum_{\xi_i \notin E}g(\xi_i)\abs{\Delta_i}} \leqslant Cn\delta + 0 = \varepsilon.
    \]
    Согласно определению интеграла Римана, $g \in R[a; b]$ и $\ds\int\limits_a^bg = 0$.
\end{proof}

\begin{proof}
    Разность $\widetilde{f} - f = \vcentcolon g$ равна нулю вне конечного множества $E \subset [a; b]$. Тогда $g \in R[a; b]$ и $\ds\int\limits_a^bg = 0$. Значит, $\widetilde{f} = f + g \in R[a; b]$ и $\ds\int\limits_a^b\widetilde{f} = \int\limits_a^bf + \int\limits_a^bg = \int\limits_a^bf$.
\end{proof}

\begin{theorem}[Достаточное условие положительности интеграла]
    Допустим, функция $f$ интегрируема по Риману и неотрицательна на отрезке $[a; b]$, а также непрерывна в точке $x_0 \in [a; b]$, в которой $f(x_0) > 0$. Тогда $\ds\int\limits_a^bf(x)dx > 0$.
\end{theorem}

\begin{proof}
    Т.\,к. $f \in C(x_0)$ и $f(x_0) > 0$, найдётся отрезок $I \subset [a; b]$ такой, что $\ds f(x) \geqslant \frac{f(x_0)}{2} > 0$ на $I$. Положим
    \[
        g(x) \vcentcolon =
        \begin{cases}
            \frac{f(x_0)}{2},& x \in I,\\
            0,& x \in [a; b] \setminus I.
        \end{cases}
    \]
    Тогда $f \geqslant g$ на $[a; b]$, отрезок $[a; b]$ есть объединение двух или трёх неперекрывающихся отрезков, один из которых есть $I$,
    \[
        \int\limits_a^bf \geqslant \int\limits_a^bg = \int\limits_Ig = \frac{f(x_0)}{2}\abs{I} > 0.
    \]
\end{proof}

\begin{theorem}[Об интеграле по симметричному отрезку от чётных и нечётных функций]
    Пусть $f \in R[-a; a]$. Если функция $f$ чётна, то
    \[
        \int\limits_{-a}^0f = \int\limits_0^af,\quad\int\limits_{-a}^af = 2\int\limits_0^af,
    \]
    а если нечётна, то
    \[
        \int\limits_{-a}^0f = -\int\limits_0^af,\quad\int\limits_{-a}^af = 0.
    \]
\end{theorem}

\begin{proof}
    Каждому отмеченному разбиению  $T\xi = \{(\Delta_i, \xi_i)\}_i$ отрезка $[0; a]$ однозначно соответствует <<симметричное>> отмеченное разбиение $\widetilde{T}$ (того же диаметра), отрезки и метки в котором симметричны $\Delta_i$ и $\xi_i$ относительно нуля. Для чётной функции $f$
    \[
        \mathcal{S}(f, T\xi) = \sum_{i = 1}^mf(\xi_i)\abs{\Delta_i} = \sum_{i = 1}^mf(-\xi_i)\abs{\Delta_i} = \mathcal{S}(f, \widetilde{T}\widetilde{\xi}),
    \]
    и определение интеграла Римана даёт $\ds\int\limits_0^af = \int\limits_{-a}^0f$. Далее,
    \[
        \int\limits_{-a}^af = \int\limits_{-a}^0f + \int\limits_0^af = 2\int\limits_0^af.
    \]
    Если $f$ нечётна, то
    \begin{gather*}
        \mathcal{S}(f, T\xi) = \sum_{i = 1}^mf(\xi_i)\abs{\Delta_i} = \sum_{i = 1}^m-f(-\xi_i)\abs{\Delta_i} = -\mathcal{S}(f, \widetilde{T}\widetilde{\xi}),\\
        \int\limits_{-a}^0f = -\int\limits_0^af,\quad\int\limits_{-a}^af = \int\limits_{-a}^0f + \int\limits_0^af = -\int\limits_0^af + \int\limits_0^af = 0.
    \end{gather*}
\end{proof}

\begin{theorem}[Об интегрируемости кусочно-непрерывных функций]
    Если функция $f$ кусочно-непрерывна на отрезке $[a; b]$, то $f \in R[a; b]$.
\end{theorem}

\begin{proof}
    См. утверждение 1 в вопросе 7.
\end{proof}

