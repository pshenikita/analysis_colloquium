\section{Колебания функции на множестве}

\begin{definition}
    \textit{Колебание} функции $f: X \to \R$ \textit{на множестве} $A \subset X$ --- величина
    \[
        \omega(f, A) \vcentcolon = \sup\limits_{x_1, x_2 \in A}\abs{f(x_1) - f(x_2)}.
    \]
\end{definition}

Если $f$ ограничена на $A$, то величина $\omega(f, A)$ конечна. В самом деле, ограниченность означает существование $C > 0$ такого, что $|f(x)| \leqslant C$ для любого $x \in A$. В этом случае
\[
    \forall x_1, x_2 \in A\;\abs{f(x_1) - f(x_2)} \leqslant |f(x_1)| + |f(x_2)| \leqslant 2C \Rightarrow w(f, A) \leqslant 2C.
\]

Напротив, если $f$ не ограничена на $A$, то $w(f, A) = +\infty$, т.\,к. можно, фиксировав $x_2$, за счёт выбора $x_1$ сделать величину $\abs{f(x_1) - f(x_2)}$ больше любого наперёд заданного $C$.

\begin{proposal}
    Если $f$ ограничена на множестве $A$, то
    \[
        \omega(f, A) = \sup\limits_{x \in A}f(x) - \inf\limits_{x \in A}f(x).\eqno(\ast)
    \]
\end{proposal}

\begin{proof}
    Возьмём произвольное $\varepsilon > 0$. Согласно определению $\sup$ и $\inf$ найдутся такие $x_1, x_2 \in A$, что
    \[
        \sup\limits_{x \in A}f(x) - \varepsilon < f(x_1),\quad f(x_2) < \inf\limits_{x \in A}f(x) + \varepsilon.
    \]
    Из определения колебания функции на множестве вытекает существование $x_3, x_4 \in A$, для которых $\omega(f, A) - \varepsilon < \abs{f(x_3) - f(x_4)}$. Не ограничивая общности, считаем $f(x_3) \geqslant f(x_4)$, так что
    \[
        \omega(f, A) - \varepsilon < f(x_3) - f(x_4).
    \]
    Получаем:
    \[
        \omega(f, A) - \varepsilon < f(x_3) - f(x_4) \leqslant \sup\limits_{x \in X}f(x) - \inf\limits_{x \in X}f(x) < f(x_1) - f(x_2) + 2\varepsilon \leqslant \omega(f, A) + 2\varepsilon,
    \]
    откуда $\omega(f, A) - \varepsilon < \sup\limits_{x \in X}f(x) - \inf\limits_{x \in X}f(x) < \omega(f, A) + 2\varepsilon$. Т.\,к. $\varepsilon > 0$ выбиралось произвольно, то имеет место $(\ast)$.
\end{proof}

\begin{proposal}[Свойства колебания]
    Пусть $\abs{f(x)} \leqslant M$ и $\abs{g(x)} \leqslant M$ для некоторого $M > 0$ и всех $x \in [a; b]$, а $\alpha, \beta \in \R$. Тогда
    \begin{enumerate}
        \begin{minipage}{.5\textwidth}
            \item $\omega(\abs{f}, A) \leqslant \omega(f, A)$;
            \item $\omega(fg, A) \leqslant M\br{\omega(f, A) + \omega(g, A)}$;
            \item $\omega(\alpha f, A) = \abs{\alpha}\omega(f, A)$;
        \end{minipage}
        \begin{minipage}[b]{.5\textwidth}
            \item $\omega(f + g, A) \leqslant \omega(f, A) + \omega(g, A)$;
            \item $\omega(\alpha f + \beta g) \leqslant \abs{\alpha}\omega(f, A) + \abs{\beta}\omega(g, A)$.
        \end{minipage}
    \end{enumerate}
\end{proposal}

\begin{proof}
    Найду силы --- напишу.
\end{proof}

