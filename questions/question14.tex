\section{Первая теорема о среднем для интеграла Римана. Преобразование Абеля}

\begin{theorem}[Первая теорема о среднем для интеграла Римана]
    Пусть:
    \begin{enumerate}
        \item $f \in B[a; b]$, $m \leqslant f(x) \leqslant M$ для всех $x \in [a; b]$;
        \item $g \in R[a; b]$ и $g(x) \geqslant 0$ для каждого $x \in [a; b]$;
        \item $fg \in R[a; b]$.
    \end{enumerate}
    Тогда
    \[
        m\int\limits_a^bg(x)dx \leqslant \int\limits_a^bf(x)g(x)dx \leqslant M\int\limits_a^bg(x)dx.\eqno(\ast)
    \]

    Если, дополнительно, $f \in C[a; b]$, то существует $c \in [a; b]$ такое, что
    \[
        \int\limits_a^bf(x)g(x)dx = f(c)\int\limits_a^bg(x)dx.\eqno(\star)
    \]
\end{theorem}

\begin{proof}
    Если $g(x) \geqslant 0$, то $mg(x) \leqslant f(x)g(x) \leqslant Mg(x)$. Интегрируя, получаем $(\ast)$.

    Докажем второе утверждение теоремы. Если интеграл $\ds\int\limits_a^bg$ равен нулю, то из $(\ast)$ видно, что $\ds\int\limits_a^bfg = 0$, и равенство $(\star)$ верно при любом $c \in [a; b]$; если не равен, поделим на него $(\ast)$:
    \[
        m \leqslant \int\limits_a^bf(x)g(x)dx \bigg/ \int\limits_a^bg(x)dx \leqslant M.
    \]
    По теореме о промежуточном значении для непрерывной функции, заключаем, что найдётся $c \in [a; b]$ такое, что
    \[
        f(c) = \int\limits_a^bf(x)g(x)dx \bigg/ \int\limits_a^bg(x)dx.
    \]
\end{proof}

Взяв $g \equiv 1$, получаем

\begin{corollary}
    Если $f \in C[a; b]$, то для некоторого $c \in [a; b]$ справедливо равенство
    \[
        \int\limits_a^bf(x)dx = f(c)(b - a).
    \]
\end{corollary}

\begin{remark}
    Формула $(\star)$ даёт следующую оценку для интеграла в её левой части:
    \[
        \abs{\int\limits_a^bf(x)g(x)dx} \leqslant \max\limits_{x \in [a; b]}\abs{f(x)}\int\limits_a^bg(x)dx.
    \]
\end{remark}

\begin{theorem}[Преобразование Абеля]
    Пусть $A_k \vcentcolon = \sum\limits_{i = 1}^ka_i$, $k = 0, 1, \ldots, n$ (при $k = 0$ пустая сумма). Тогда
    \[
        \sum_{i = 1}^na_ib_i = A_nb_n + \sum_{i = 1}^{n - 1}A_i(b_i - b_{i + 1}).
    \]
\end{theorem}

\begin{proof}
    В самом деле,
    \begin{multline*}
        \sum_{i = 1}^na_ib_i = \sum_{i = 1}^n(A_i - A_{i - 1})b_i = \sum_{i = 1}^nA_ib_i - \sum_{i = 1}^nA_{i - 1}b_i = \sum_{i = 1}^nA_ib_i - \sum_{i = 0}^{n - 1}A_ib_{i + 1} =\\ = A_nb_n - A_0b_1 + \sum_{i = 1}^{n - 1}A_i(b_i - b_{i + 1}) = A_nb_n + \sum_{i = 1}^{n - 1}A_i(b_i - b_{i + 1}).
    \end{multline*}
\end{proof}

